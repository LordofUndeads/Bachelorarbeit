\section{Ausblick und Zusammenfassung}
Wie bereits im Kapitl über die Praktische Implementierung angedeutet gibt es einige Punkte, welche bis zum Zeitpunkt der Abgabe nicht umgesetzt wurden.
Diee seien hier als mögliche Erweiterungen und geplante Ergänzungen aufgeführt.

Allen Punkten voran steht die Implementierung der manuellen Auswahl von Dreiecken als Heuristik. Diese wurde Eingangs als Schwerpunkt genannt, welcher in dieser Arbeit zu erreihen sein soll. 
Aus zeitlichen Gründen, welche in der Auswertung genauer beleuchtet wurden, war es allerdings nicht möglich, diesen Punkt zu inkludieren. Geplant war es, dass die Nutzerauswahlen gespeichert, 
ausgewertet und in eine neue Heuristik überführt würden. Dies ist eine der größten möglichen Erweiterungen dieses Programms, da dadurch die Interaktivität dieser Software um ein Vielfaches 
ansteigen würde.

Auch ist es bisher noch nicht gelungen, die \ac{dt} als anwählbare Trinagulationsmethode einzubauen, geschweige denn sie als Vergleich auf der Ergebnisseite 
anzubieten. Dies ist ein Punkt, welcher nicht im unmittelbaren Fokus der Arbeit lag, da der \ac{eca} explizit als Algorithmus angestrebt war.
Dennoch ist die Erweiterung des Programms, nicht nur um die \ac{dt} sondern auch um andere Triangulationsalgorithem, durchaus denkbar und wünschenswert.

Bei Erweiterungen des Programms kommen auch weitere zusätzliche Heuristiken in Frage. Da auch ein Vergleich der Einflüsse von verschiedenen Heuristiken auf die Triangulationsqualität ein Ziel dieser 
Arbeit war, liegt es nur nahe weitere solche Heuristiken einzubauen.

Auch an möglichen Optionen für die Triangulation und dem Erscheinungsbild der Anwendung besteht reichhaltiges Erweiterungspotential. Neben der in den theoretischen Grundlagen angesprochenen Methode des 
Edge Swappings sind auch andere Optionen wie beispielsweie ein unterer Grenzwert für den kleinsten Innenwinkel der Dreiecke denkbar. Liegt ein Innenwinkel unter desem Wert, könnte der Algorithmus dieses dann 
vorläufig ignorieren und erst dann Auswählen, wenn keine besseren Optionen zur Verfügung stehen.

Aber auch an optischen Einstellungsmöglichkeiten könnte in ZUkunft noch einiges optimiert und hinzugefügt werden. Neben dem beireits implementierten Dark Mode können auch andere Modi für bessere Lesbarkeit bei Farbenblindheit 
eingebaut werden. Auch einen Sprachauswahl ist denkbar, da die Software vorläufig nur in der englischen Sprache verfasst ist. Diese würde die Barrierefreiheit stark erhöhen, da es dann nichtmehr erforderlich wäre, des Englischen mächtig sein 
zu müssen.

Bei optischen Aspekten, welche die Nutzerfreundlichkeit erhöhen, kommen auch methaphorische Icons in den Sinn. An Stelle von Beschriftungen der Buttons könnten dann aussagekräftige Bilder stehen.
Ein Stift für die Zeichenoption, eine Mülltonne für das Löschen und ähnliches, was auch bereit in anderen Zeichenprogrammen gängig ist. Der Zugang des Nutzers würde dann steigen, sollte er zuvor einmal ein anderes Zeichenprogramm benutzt haben.
Es ist also noch einiges an Erweiterungen für die, in dieser Arbeit entworfenen, Software denkbar. Einiges davon wird voraussichtlich noch nach Abgabe dieser Arbeit erreicht werden, anderes nicht. \linebreak

Zusammenfassend gibt es also noch zu sagen, dass auf Grund zeitlicher Probleme nicht all die angedachten Features des Programms umgesetzt wurden. Das \ac{gui} allerdings ist ohne Fehler lauffähig.
Auch ist es möglich eine Triangulation, wie angestrebt, durchzuführen und dies schrittweise. Damit ist einer der Hauptaspekte dieser Arbeit erfüllt. Die Anschaulichkeit einer Triangulation ist umgesetzt und kann beispielsweie
in der Lehre verwendet werden.
An anderer Stelle gibt es noch Verbesserungsbedarf, dies wird wie angesprochen auch teilweise noch geschehen. Es ist jedoch auch abzusehen, dass eine solche Software ständig erweiterbar ist und demnach keine wirkliche völlständige Implementierung erziehlt 
werden kann. Allein die Heuristiken für die Auswahl des nächsten Dreiecks sind nahezu endlos. 